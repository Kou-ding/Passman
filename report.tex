% Document class and two-column conversion
\documentclass{report}
% dimensions of paper and relative text positioning
\usepackage[a4paper,top=2cm,bottom=2cm,left=2cm,right=2cm]{geometry}
% math symbols
\usepackage{amsmath}
\usepackage{amssymb}
% package for including URLs
\usepackage{url}
% Required for including images
\usepackage{graphicx}
\usepackage{float} % Required for specifying the exact location of a figure

\usepackage{minted} % Required for including code snippets

% enable writing in greek
\usepackage[greek,english]{babel}
\usepackage[utf8]{inputenc}

\setlength{\parindent}{0pt} % Removes all indentation from paragraphs

% Start of the document
\begin{document}

% Set the language to greek
\selectlanguage{greek}

% Title page
\title{\Huge \bfseries \selectlanguage{english} Passman Project 2024\selectlanguage{greek}} %\Huge and \bfseries are used to make the title bigger and bold
\author{Παπαδάκης Κωνσταντίνος Φώτιος\vspace{0.5cm} \\  ΑΕΜ:10371} % \vspace{0.5cm} is used to add some vertical space between the author and the AEM
\date{\today}
% prints the title, author and date on a separate page
\maketitle

\section*{Περιγραφή προβλήματος}
Η εφαρμογή $Passman$ αποτελεί μια απλουστευμένη υλοποίηση διαχειριστή κωδικών. Στα πλαίσια της εργασίας
μας καλούμαστε να επισημάνουμε τα κενά ασφαλείας της εν λόγω εφαρμογής, να παρουσιάσουμε ευπάθειες, να
προτείνουμε τρόπους αντιμετώπισης και να εφαρμόσουμε τις απαραίτητες αλλαγές, όπου αυτό είναι δυνατόν.

\section*{Περιγραφή διορθώσεων}
Για να ανταπεξέλθουμε στις απαιτήσεις της εργασίες επιστρατεύσαμε διάφορες τακτικές τις οποίες θα 
εξηγήσουμε στη συνέχεια.
\begin{itemize}
    \item \textbf{Περιορισμοί χρηστών βάσης:} Συνδεόμαστε στη βάση ως ένας χρήστης με περιορισμένα 
    δικαιώματα.
    \item \textbf{Κρυπτογράφηση κωδικών:} Χρησιμοποιούμε $hash$ αντί να παραθέτουμε τους κωδικούς 
    στη βάση ως απλό κείμενο.
    \item \textbf{Παραμετροποιημένα ερωτήματα:} Προστασία από $SQL$ $Injections$.
    \item \textbf{Καθαρισμός εισόδου:} Προστασία από $XSS$ επιθέσεις.
\end{itemize}

\section*{Παραδείγματα εκμετάλλευσης ευπαθειών}
Παράδειγμα επίθεσης $XSS$ υπό την έλλειψη καθαρισμού της εισόδου:
\selectlanguage{english}
\begin{minted}{html}
    <!-- For user input on the dashboard page such as: -->
    <script>alert('XSS Attack example!');</script>
    <!-- we can execute javascript code -->
    <!-- Instead now, after sanitization: -->
    $new_website = htmlspecialchars(trim($_POST["new_website"]), ENT_QUOTES, 'UTF-8');
    <!-- it is going to be displayed as the following text: -->
    <!-- &lt;script&gt;alert('XSS Attack example!')&lt;/script&gt; -->
\end{minted}
\selectlanguage{greek}
Παράδειγμα επίθεσης $SQL$ $Injection$ χωρίς παραμετροποίηση:
\selectlanguage{english}
\begin{minted}{sql}
    SELECT * FROM login_users WHERE username = 'admin' OR '1' = '1';
    -- This will return all the users in the database
    -- Can be avoided through proper parameterization
\end{minted}
\selectlanguage{greek}
Παράδειγμα εκμετάλλευσης πρακτικής μη κρυπτογράφησης κωδικών:
\selectlanguage{english}
\begin{minted}{sql}
    SELECT * FROM login_users;
    -- This will return all the users and their passwords in the database
    -- Can be avoided by hashing the passwords before storing them
\end{minted}
\selectlanguage{greek}
Παράδειγμα επίθεσης χρήστη με όλα τα δικαιώματα:
\selectlanguage{english}
\begin{minted}{sql}
    DROP DATABASE pwd_mgr;
    -- This will delete the entire database and all its data
    -- Can be avoided by using a user with limited privileges
\end{minted}
\selectlanguage{greek}
\section*{Περιγραφή Αλλαγών Κώδικα}
Στη συνέχεια πρόκειται να παραθέσουμε τον παλιό μαζί με τον καινούριο κώδικα εστιάζοντας στις 
αλλαγές που πραγματοποιήσαμε καθώς και το νόημα αυτών.
% Database connection
\subsection*{$connection.php$}
Εδώ προδιαγράφουμε τη σύνδεση με τη βάση δεδομένων μας. 
\selectlanguage{english}
\begin{minted}{php}
    $servername = "localhost";
    $username = "limited_user"; // Use a non-admin user
    $password = "limited_password";
    $dbname = "pwd_mgr";

    // Create connection
    $conn = new mysqli($servername, $username, $password, $dbname);
\end{minted}
\selectlanguage{greek}
Έτσι οι χρήστες αλληλεπιδρούν με τη βάση με τα προνόμια του χρήστη $limited\_user$ και όχι του $root$. 
Η δημιουργία του χρήστη $limited\_user$ και η ανάθεση των $privileges$ του γίνεται ως εξής:
\selectlanguage{english}
\begin{minted}{sql}
    CREATE USER 'limited_user'@'localhost' IDENTIFIED BY 'limited_password';
    GRANT SELECT, INSERT, UPDATE, DELETE ON pwd_mgr.* TO 'limited_user'@'localhost';
\end{minted}
\selectlanguage{greek}
Του δίνουμε πρόσβαση μόνο στη βάση $pwd\_mgr$ και μόνο για τις $SELECT$, $INSERT$, $UPDATE$ και 
$DELETE$ ενέργειες.
% Register passman users
\subsection*{$register.php$}
Η πρώτη αλλαγή στο συγκεκριμένο αρχείο απαιτεί τον καθαρισμό του ονόματος χρήστη και του κωδικού
από ειδικούς χαρακτήρες στην αρχή και το τέλος του αλφαριθμητικού κατά την εισαγωγή τους στη βάση.
\selectlanguage{english}
\begin{minted}{php}
    // Get user submitted information
    $new_username = trim($_POST['new_username']);
    $new_password = trim($_POST['new_password']);
\end{minted}
\selectlanguage{greek}
Σημαντικό είναι να σημειωθεί ότι στο ενδιάμεσο γίνεται και η κρυπτογράφηση του κωδικού έτσι ώστε να 
μην μπορεί να αξιοποιηθεί από κάποιον επιτιθέμενο η πληροφορία αυτή.
\selectlanguage{english}
\begin{minted}{php}
    $hashed_password = password_hash($new_password, PASSWORD_BCRYPT);
\end{minted}
\selectlanguage{greek}
Στη συνέχεια προετοιμάζουμε το παραμετροποιημένο ερώτημά μας προστατεύοντας από $SQL$ $Injections$. 
Ουσιαστικά αυτό καθιστά αδύνατον να ερμηνευτεί η είσοδος του χρήστη ως $SQL$ εντολή.
\selectlanguage{english}
\begin{minted}{php}
    $stmt = $conn->prepare("INSERT INTO login_users (username, password) VALUES (?, ?)");
    $stmt->bind_param("ss", $new_username, $hashed_password);
\end{minted}
\selectlanguage{greek}
% Login passman users
\subsection*{$login.php$}
Ομοίως, και εδώ καθαρίζουμε και παραμετροποιούμε το ερώτημά μας ενώ παράλληλα επαληθεύουμε
τον κωδικό του χρήστη μέσω του $hash$ του.\\
\textbf{Καθαρισμός εισόδου:}
\selectlanguage{english}
\begin{minted}{php}
    // Get user submitted information
    $username = trim($_POST['username']);
    $password = trim($_POST['password']);
\end{minted}
\selectlanguage{greek}
\textbf{Παραμετροποίηση:}
\selectlanguage{english}
\begin{minted}{php}
    // Prepare an SQL query
    $stmt = $conn->prepare("SELECT password FROM login_users WHERE username = ?");
    $stmt->bind_param("s", $username);
    $stmt->execute();
    $stmt->store_result();
\end{minted}
\selectlanguage{greek}
\textbf{Επαλήθευση κωδικού:}
\selectlanguage{english}
\begin{minted}{php}
    if ($stmt->num_rows > 0) {
        $stmt->bind_result($hashed_password);
        $stmt->fetch();

        if (password_verify($password, $hashed_password)) {
            $_SESSION['username'] = $username;
            $_SESSION['loggedin'] = true;
            header("Location: dashboard.php");
            exit;
        } else {
            $login_message = "Invalid username or password";
        }
    } else {
        $login_message = "Invalid username or password";
    }
\end{minted}
\selectlanguage{greek}
% Dashboard
\subsection*{$dashboard.php$}
Στον κώδικα της κεντρικής σελίδας καθαρίζουμε την είσοδο του χρήστη πριν την αποθηκεύσουμε στη βάση.
Επειδή πρόκειται να προβάλουμε τα στοιχεία αυτά στην σελίδα μέσω $html$ είναι απαραίτητο να τα περάσουμε 
και από την συνάρτηση $htmlspecialchars$ για να αποφύγουμε τυχόν επιθέσεις $XSS$ όπου εκτελείται κώδικας
$javascript$ μέσα στη σελίδα:
\selectlanguage{english}
\begin{minted}{php}
    $new_website = htmlspecialchars(trim($_POST["new_website"]), ENT_QUOTES, 'UTF-8');
    $new_username = htmlspecialchars(trim($_POST["new_username"]), ENT_QUOTES, 'UTF-8');
    $new_password = htmlspecialchars(trim($_POST["new_password"]), ENT_QUOTES, 'UTF-8');
\end{minted}
\selectlanguage{greek}
Παραμετροποιούμε το ερώτημα:
\selectlanguage{english}
\begin{minted}{php}
    $stmt = $conn->prepare("INSERT INTO websites (login_user_id, web_url, web_username, web_password) VALUES ((SELECT id FROM login_users WHERE username = ?), ?, ?, ?)");
    $stmt->bind_param("ssss", $username, $new_website, $new_username, $new_password);
\end{minted}
\selectlanguage{greek}
Ομοίως για τα ερωτήματα $DELETE$ και $SELECT$.
% Passman user notes
\subsection*{$notes.php$}
Τελευταία έχουμε το αρχείο υπεύθυνο για την διαχείρηση σημειώσεων από τον χρήστη. Επαναλαμβάνονται 
οι ίδιες τεχνικές που είδαμε παραπάνω:\\
\textbf{Καθαρισμός εισόδου:}
\selectlanguage{english}
\begin{minted}{php}
    $new_note = htmlspecialchars(trim($_POST["new_note"]), ENT_QUOTES, 'UTF-8');
\end{minted}
\selectlanguage{greek}
\textbf{Παραμετροποίηση:}
\selectlanguage{english}
\begin{minted}{php}
    $stmt = $conn->prepare("INSERT INTO notes (login_user_id, note) 
                            VALUES ((SELECT id FROM login_users WHERE username = ?), ?)");
    $stmt->bind_param("ss", $username, $new_note);
\end{minted}
\selectlanguage{greek}

\end{document}